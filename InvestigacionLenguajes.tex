% !TEX TS-program = pdflatex
% !TEX encoding = UTF-8 Unicode

% This is a simple template for a LaTeX document using the "article" class.
% See "book", "report", "letter" for other types of document.

\documentclass[11pt]{article} % use larger type; default would be 10pt

\usepackage[utf8]{inputenc} % set input encoding (not needed with XeLaTeX)

%%% Examples of Article customizations
% These packages are optional, depending whether you want the features they provide.
% See the LaTeX Companion or other references for full information.

%%% PAGE DIMENSIONS
\usepackage{geometry} % to change the page dimensions
\geometry{a4paper} % or letterpaper (US) or a5paper or....
% \geometry{margin=2in} % for example, change the margins to 2 inches all round
% \geometry{landscape} % set up the page for landscape
%   read geometry.pdf for detailed page layout information

\usepackage{graphicx} % support the \includegraphics command and options

% \usepackage[parfill]{parskip} % Activate to begin paragraphs with an empty line rather than an indent

%%% PACKAGES
\usepackage{booktabs} % for much better looking tables
\usepackage{array} % for better arrays (eg matrices) in maths
%\usepackage{paralist} % very flexible & customisable lists (eg. enumerate/itemize, etc.)
\usepackage{xcolor} % adds environment for commenting out blocks of text & for better verbatim
\usepackage{subfig} % make it possible to include more than one captioned figure/table in a single float
% These packages are all incorporated in the memoir class to one degree or another...

%%% HEADERS & FOOTERS
\usepackage{fancyhdr} % This should be set AFTER setting up the page geometry
\pagestyle{fancy} % options: empty , plain , fancy
\renewcommand{\headrulewidth}{0pt} % customise the layout...
\lhead{}\chead{}\rhead{}
\lfoot{}\cfoot{\thepage}\rfoot{}

%%% SECTION TITLE APPEARANCE
\usepackage{sectsty}
\allsectionsfont{\sffamily\mdseries\upshape} % (See the fntguide.pdf for font help)
% (This matches ConTeXt defaults)

%%% ToC (table of contents) APPEARANCE
\usepackage[nottoc,notlof,notlot]{tocbibind} % Put the bibliography in the ToC
\usepackage[titles,subfigure]{tocloft} % Alter the style of the Table of Contents
\renewcommand{\cftsecfont}{\rmfamily\mdseries\upshape}
\renewcommand{\cftsecpagefont}{\rmfamily\mdseries\upshape} % No bold!

%%% END Article customizations

\usepackage[spanish]{babel}
\usepackage{listings} 
%%% The "real" document content comes below...


\title{Investigación de Lenguajes - C\#}
\author{Erick Buendia}
%\date{} % Activate to display a given date or no date (if empty),
         % otherwise the current date is printed 

\begin{document}
\maketitle
%\tableofcontents % No hace falta un TOC en un artículo corto

\section{Introducción}


\section{Características}

 \begin{enumerate}
 \item Orientacion a objetos de forma simple, moderno y de propósito general.
 \item Provee principios de Ingenieria de Software como:
	\begin{enumerate}
 		\item Revisión estricta de los tipos de datos
 		\item Revisión de límites de vectores
		\item Detección de intentos de usar variables no inicializadas
		\item Recolección de basura automática.
	 \end{enumerate}
\item Eficiente ya que todo el código incluye numerosas restricciones para garantizar su seguridad.
\item Desarrollo de componentes para ambientes distribuidos.
\item Minimo consumo de memoria y almacenamiento
\item Codigo fuente portable
\item Lenguaje autocontenido

\end{enumerate} 

\section{Compiladores}

\begin{itemize}
 \item Principal IDE:  Microsoft Visual Studio
 \item Microsoft .Net Framework 2.0
 \item SharpDevelop
 \item Delphi 2006
\end{itemize} 

\section{Historia}
\paragraph{ }
En sus inicios este lenguaje tenia el nombre de Cool (Lenguaje C Orientado a Objetos) y fue desarrollado y estandarizado por Microsoft como parte de su plataforma .NET en 2001. Este lenguaje no incorpora el uso de punteros; debido a que fue desarrollado desde cero se modifiacron algunas prestaciones.
\paragraph{ }
La sintaxis de este lenguaje deriva de C/C++ y utiliza el modelo de objetos de la plataforma .NET, similar al de Java, aunque incluye mejoras derivadas de otros lenguajes.
\paragraph{ }
Su nombre C Sharp viene inspirado de una nota musical (do sostenido) con un semitono mas alto que da a entender que es superior a  C/C++ y es considerado un lenguaje multiplataforma.

\section{Tutorial de Instalación}


\section{Ejemplos Básicos}

\lstdefinestyle{sharpc}{language=[Sharp]C, frame=lr, rulecolor=\color{blue!80!black}}

Fig 5.1: Ejemplo básico "Hello World"
\lstset{style=sharpc}
\begin{lstlisting}[frame=single]
using System;

public class Ejemplo1{
	public static void Main(string[] arg){
		Console.WriteLine("Hello World");
	}
}
\end{lstlisting}

Fig 5.2: Suma, Resta, Multiplicación y División entre dos números enteros
\begin{lstlisting}[frame=single]
using System;

public class Ejemplo2{
	public static void Main(string[] arg){
		int n1;	//Variable para almacenar el primer numero
		int n2;	//Variable para almacenar el segundo numero
		
		//Obtenemos el primer numero
		Console.WriteLine("Introduzca el primer numero:");
		Console.ReadLine(n1);
		
		//Obtenemos el segundo numero
		Console.WriteLine("Introduzca el segundo numero:");		
		Console.ReadLine(n2);
		
		//Muestra la suma
		Console.WriteLine("Suma: {0}", n1 + n2);
		
		//Muestra la resta		
		Console.WriteLine("Resta: {0}", n1 - n2);
		
		//Muestra el producto		
		Console.WriteLine("Multiplicacion: {0}", n1 * n2);
		
		//Muestra la division		
		Console.WriteLine("Division: {0}", n1 / n2);
	}
}
\end{lstlisting}

\end{document}
