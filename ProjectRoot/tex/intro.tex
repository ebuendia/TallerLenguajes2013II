\documentclass[../main.tex]{subfiles}
\begin{document}
\section{Introducción}
Amado por muchos, eludido por otros. En éste artículo vamos a tratar ciertos puntos de un lenguaje en particular, el cual ha tenido un desarrollo bastante interesante. Dicho lenguaje es lenguaje C\#.

Éste, es un lenguaje de programación orientado a objetos que pertenece al framework .NET de Microsoft.
C\# es un \textit{lenguaje “compilado''}, que, aunque es una expresión un tanto informal, suele a hacer referencia al hecho de que una vez que se ha escrito el código fuente en determinado lenguaje, por medio del compilador, éste se traduce para que pueda ser ejecutado en una plataforma determinada (ya sea Windows, Mac, Linux, etc). 

Una desventaja bastante notable respecto a los \textit{lenguajes “interpretados''} es la portabilidad, debido a que en el caso de los lenguajes interpretados, el código escrito en éstos es traducido a un código intermedio (bytecode en Java) el cual puede ser ejecutado en una máquina virtual, independientemente de la plataforma que se use. 

Comúmente, C\# y Java son comparados, lo cual genera debates, sin embargo la intención de este artículo no es fomentar éste hecho\footnote{www.google.com}.

Los temas a tratar son: el origen y evolución de C\#, características del dicho lenguaje,se desarrollarán algunos tutoriales de instalación y algunos proyectos sencillos.








\clearpage
\end{document}
